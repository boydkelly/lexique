\documentclass[a4paper,12pt]{article}
\usepackage{tabularx,enumitem,ragged2e,tex4ebook}
\usepackage{fontspec}
\setmainfont{arimo}
\usepackage[margin=25mm,showframe]{geometry}
% Automatic line breaking and hanging indentation in 1st col.:
\newcolumntype{P}[1]{>{\RaggedRight\arraybackslash%
     \hangafter=1\hangindent=1em}p{#1}}
% Define custom enumerated and itemized list environments:
\newlist{definition}{enumerate}{1}
\setlist[definition,1]{nosep,wide=0pt,label=\arabic*),
                        before=\begin{minipage}[t]{\hsize},
after =\end{minipage}\vspace{10pt}}
\newlist{examples}{itemize}{1}
\setlist[examples,1]{label=-,after=\vspace{20pt}}

\begin{document}
\noindent
\begin{tabularx}{\textwidth}{@{} P{5cm} X @{}}
		a/aw &
\begin{definition}
\item vous pr. 2iè pl. 
  \begin{examples}
  \item a bamuso nana

		Votre mère est venue
  \item n k'aw ye.

  Je vous ai vus.
  \end{examples} 
\item Second translated definition 
  \begin{examples}
  \item bla bla bla

		This is the meaning
  \item ble ble ble 
  \end{examples}
\end{definition} 
\\
a &
\begin{definition}
\item il/elle pr. 3iè s.
	\begin{examples}
	\item Awa be min ? a be bon kɔnɔ.

		Où est Awa ? Elle est dans la maison. 
	\item Musa k'a dɔgcɛ bugɔ.

		Moussa a frappé son frère.
	\item i ta mobili, n m'a ye fɔlɔ.

		Ta voiture, je ne l'ai pas encore vue.  
	\end{examples}
\end{definition}
\\
báara &
\begin{definition}
\item travail
	\begin{examples}
	\item a be baara kɛ.

		I travaille.
	\item a be barra la.

		Il est au travail.
	\end{examples}
  \end{definition}
	\begin{definition}
	\item travailler ( au matériau )
		\begin{examples}
		\item a be nɛgɛ baara.

			Il travaille le fer.
		\end{examples}	
	\end{definition}
\end{tabularx}

\end{document} 
